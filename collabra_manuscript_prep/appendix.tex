

\hypertarget{contributions}{%
\section{Contributions}\label{contributions}}

Contributed to Conception and design: CMH, MMG, ERI, ELK

Contributed to acquisition of data: CMH

Contributed to analysis and interpretation of data: CMH, MMG, ERI, ELK

Drafted and/or revised the article: CMH, MMG, ERI, ELK

Approved the submitted version for publication: CMH, MMG, ERI, ELK

\hypertarget{funding-information}{%
\section{Funding Information}\label{funding-information}}

The Study reported here was funded by University of Limerick, Education
and Health Sciences seed funding. Study S5 was funded by Mary Immaculate
College seed funding.

\hypertarget{data-accessibility-statement}{%
\section{Data Accessibility
Statement}\label{data-accessibility-statement}}

All data and analysis code are publicly available on this project's OSF
page at
\url{https://osf.io/fcd5r/?view_only=9fb6e506e53340c189b98453bb2b6eaf}.
Materials are also available including the full text of the jsPsych
script.

\hypertarget{figure-titles}{%
\section{Figure Titles}\label{figure-titles}}

\hypertarget{main-manuscript}{%
\subsection{Main Manuscript}\label{main-manuscript}}

Figure 1: Hypothesized relationship between deliberation and responses
in the dumbfounding paradigm

Figure 2: Responses to critical slide depending on cognitive load

Figure 3: Responses to critical slide and for the experimental group and
the control group for each scenario

\hypertarget{supplementary-materials}{%
\subsection{Supplementary Materials}\label{supplementary-materials}}

Figure 1: Screenshot of Attention Check

Figure 2: Screenshot of Attention Check

Figure 3: Study S1: Responses to critical slide and for the experimental
group (\emph{N} = 33) and the control group (\emph{N} = 33)

Figure 4: Study S1: Probability of selecting each response to the
critical slide depending on Need for Cognition

Figure 5: Sample dot patterns - more simple for the control group (a)
and higher complexity for the experimental condition (b)

Figure 6: Study S2: Responses to critical slide for (left) the
experimental group (\emph{N} = 51) vs the control group (\emph{N} = 49);
and (right) depending on engagement (\emph{N} = 56) or non-engagement
(\emph{N} = 44) with the memory task

Figure 7: Study S2: Probability of selecting each response to the
critical slide depending on Need for Cognition

Figure 8: Study S3: Responses to critical slide for the cognitive load
group (\emph{N} = 68) and the control group (\emph{N} = 61)

Figure 9: Study S3: Probability of selecting each response to the
critical slide depending on Need for Cognition

Figure 10: Study S4: Responses to critical slide for the cognitive load
group (\emph{N} = 64) and the control group (\emph{N} = 61)

Figure 11: Study S4: Probability of selecting each response to the
critical slide depending on Need for Cognition

Figure 12: Study S5: Responses to critical slide and for the
experimental group (\emph{N} = 98) and the control group (\emph{N} =
106)

Figure 13: Study S5: Probability of selecting each response to the
critical slide depending on Need for Cognition
