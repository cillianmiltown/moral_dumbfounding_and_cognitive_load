% Options for packages loaded elsewhere
\PassOptionsToPackage{unicode}{hyperref}
\PassOptionsToPackage{hyphens}{url}
%
\documentclass[
]{article}
\usepackage{lmodern}
\usepackage{amssymb,amsmath}
\usepackage{ifxetex,ifluatex}
\ifnum 0\ifxetex 1\fi\ifluatex 1\fi=0 % if pdftex
  \usepackage[T1]{fontenc}
  \usepackage[utf8]{inputenc}
  \usepackage{textcomp} % provide euro and other symbols
\else % if luatex or xetex
  \usepackage{unicode-math}
  \defaultfontfeatures{Scale=MatchLowercase}
  \defaultfontfeatures[\rmfamily]{Ligatures=TeX,Scale=1}
\fi
% Use upquote if available, for straight quotes in verbatim environments
\IfFileExists{upquote.sty}{\usepackage{upquote}}{}
\IfFileExists{microtype.sty}{% use microtype if available
  \usepackage[]{microtype}
  \UseMicrotypeSet[protrusion]{basicmath} % disable protrusion for tt fonts
}{}
\makeatletter
\@ifundefined{KOMAClassName}{% if non-KOMA class
  \IfFileExists{parskip.sty}{%
    \usepackage{parskip}
  }{% else
    \setlength{\parindent}{0pt}
    \setlength{\parskip}{6pt plus 2pt minus 1pt}}
}{% if KOMA class
  \KOMAoptions{parskip=half}}
\makeatother
\usepackage{xcolor}
\IfFileExists{xurl.sty}{\usepackage{xurl}}{} % add URL line breaks if available
\IfFileExists{bookmark.sty}{\usepackage{bookmark}}{\usepackage{hyperref}}
\hypersetup{
  hidelinks,
  pdfcreator={LaTeX via pandoc}}
\urlstyle{same} % disable monospaced font for URLs
\usepackage[margin=1in]{geometry}
\usepackage{graphicx,grffile}
\makeatletter
\def\maxwidth{\ifdim\Gin@nat@width>\linewidth\linewidth\else\Gin@nat@width\fi}
\def\maxheight{\ifdim\Gin@nat@height>\textheight\textheight\else\Gin@nat@height\fi}
\makeatother
% Scale images if necessary, so that they will not overflow the page
% margins by default, and it is still possible to overwrite the defaults
% using explicit options in \includegraphics[width, height, ...]{}
\setkeys{Gin}{width=\maxwidth,height=\maxheight,keepaspectratio}
% Set default figure placement to htbp
\makeatletter
\def\fps@figure{htbp}
\makeatother
\setlength{\emergencystretch}{3em} % prevent overfull lines
\providecommand{\tightlist}{%
  \setlength{\itemsep}{0pt}\setlength{\parskip}{0pt}}
\setcounter{secnumdepth}{-\maxdimen} % remove section numbering

\author{}
\date{\vspace{-2.5em}}

\begin{document}

\hypertarget{contributions}{%
\section{Contributions}\label{contributions}}

Contributed to Conception and design: CMH, MMG, ERI, ELK

Contributed to acquisition of data: CMH

Contributed to analysis and interpretation of data: CMH, MMG, ERI, ELK

Drafted and/or revised the article: CMH, MMG, ERI, ELK

Approved the submitted version for publication: CMH, MMG, ERI, ELK

\hypertarget{funding-information}{%
\section{Funding Information}\label{funding-information}}

The Study reported here was funded by University of Limerick, Education
and Health Sciences seed funding. Study S5 was funded by Mary Immaculate
College seed funding.

\hypertarget{data-accessibility-statement}{%
\section{Data Accessibility
Statement}\label{data-accessibility-statement}}

All data and analysis code are publicly available on this project's OSF
page at
\url{https://osf.io/fcd5r/?view_only=9fb6e506e53340c189b98453bb2b6eaf}.
Materials are also available including the full text of the jsPsych
script.

\hypertarget{figure-titles}{%
\section{Figure Titles}\label{figure-titles}}

\hypertarget{main-manuscript}{%
\subsection{Main Manuscript}\label{main-manuscript}}

Figure 1: Hypothesized relationship between deliberation and responses
in the dumbfounding paradigm

Figure 2: Responses to critical slide depending on cognitive load

Figure 3: Responses to critical slide and for the experimental group and
the control group for each scenario

\hypertarget{supplementary-materials}{%
\subsection{Supplementary Materials}\label{supplementary-materials}}

Figure 1: Screenshot of Attention Check

Figure 2: Screenshot of Attention Check

Figure 3: Study S1: Responses to critical slide and for the experimental
group (\emph{N} = 33) and the control group (\emph{N} = 33)

Figure 4: Study S1: Probability of selecting each response to the
critical slide depending on Need for Cognition

Figure 5: Sample dot patterns - more simple for the control group (a)
and higher complexity for the experimental condition (b)

Figure 6: Study S2: Responses to critical slide for (left) the
experimental group (\emph{N} = 51) vs the control group (\emph{N} = 49);
and (right) depending on engagement (\emph{N} = 56) or non-engagement
(\emph{N} = 44) with the memory task

Figure 7: Study S2: Probability of selecting each response to the
critical slide depending on Need for Cognition

Figure 8: Study S3: Responses to critical slide for the cognitive load
group (\emph{N} = 68) and the control group (\emph{N} = 61)

Figure 9: Study S3: Probability of selecting each response to the
critical slide depending on Need for Cognition

Figure 10: Study S4: Responses to critical slide for the cognitive load
group (\emph{N} = 64) and the control group (\emph{N} = 61)

Figure 11: Study S4: Probability of selecting each response to the
critical slide depending on Need for Cognition

Figure 12: Study S5: Responses to critical slide and for the
experimental group (\emph{N} = 98) and the control group (\emph{N} =
106)

Figure 13: Study S5: Probability of selecting each response to the
critical slide depending on Need for Cognition

\end{document}
